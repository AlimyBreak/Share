
\documentclass{article}
\usepackage[UTF8]{ctex}

\title{测试\\换行\& 与符号}%文档标题
\author{白菜}%作者
\begin{document}%开始文章书写
\maketitle%将封面打印出来,也就是上面的标题作者
\tableofcontents %自动目录
%摘要
\vspace{8pt} %字号大小
\renewcommand{\abstractname}{\huge 摘\quad 要}
\begin{abstract} %开始写摘要
\normalsize
\noindent 这个是摘要啦
\end{abstract}
\section{第一个章节}
\paragraph{主要内容}~{} \newline \indent 换一行写正文,两个缩进
\subparagraph{子段}本次汇报主要总结最近精看的几篇深度学习文章以及关于卷积RBM在运动捕捉数据中的实验结果和进一步将展开的工作。深度学习文章主要有四篇,分别介绍了CNN在运动捕捉数据建模中的应用,以及卷积RBM最开始的提出和在Audio分类中的应用。另外一篇论文介绍了运动拼接的一种方法。
 
\section{第二个章节}
\paragraph{} 写什么好呢
\subsection{来个子章节}可以可以
 
\section{第三个章节}
\paragraph{}写一个段落
\subparagraph{}再来一个字段落
\end{document}
